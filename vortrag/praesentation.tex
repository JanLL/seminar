\documentclass{beamer}

\usepackage[ngerman]{babel}
\usepackage[utf8x]{inputenc}
\usepackage{amsmath,amsfonts,amssymb}

\usepackage[absolute,overlay]{textpos}

\newcommand{\todo}[1]{\textcolor{red}{TODO: #1}}


\usetheme{Luebeck}
\usecolortheme{orchid}

\title{Seminar Kombinatorische Optimierung: \\ Markov Random Fields}
\author{Jan Lammel}

\begin{document}
	
\frame{\titlepage}

\frame{
	\frametitle{Table of contents}
	\tableofcontents[hideallsubsections]
	
}

\section{Introduction}
\frame{
\frametitle{Introduction}

\begin{textblock}{12}(1,5)
	Until now: \\
	\begin{itemize}
	\item 2 label segmentation (foreground / background)
	\item Efficient via max-flow / min-cut
	\end{itemize}
	
	Now: \\
	\begin{itemize}
	\item Segmentation with arbitrary number of labels
	\item Actual problem is NP-complete
	\item Reduction to small 2-label problems with approximative methods
	\end{itemize}
\end{textblock}
}

\frame{
\frametitle{Introduction}
% Bild + fertige Segmentierung als Motivation was am Ende rauskommen soll.
blubb


}

\section{Mathematical problem formulation}
\subsection{Basic definitions}
\frame{
\frametitle{Basic definitions}

\begin{textblock}{14}(1, 4.5)

	\begin{itemize}
		\item Image consists of pixels $P = \{ 1, 2, ..., m \}$
		\item Segmentation labels $\mathcal{L} = \{l_1,l_ 2, ..., l_k\}$
		\item Each pixel has two statistical values
		\begin{itemize}
			\item hypothesis $f \in \mathcal{L}^m$: \ segmentation label
			\item observation $O$: \qquad \ information in the image (intensities, ...)
		\end{itemize}
	\end{itemize}
	
	\todo{Fehlt noch was wichtiges?!}

\end{textblock}

}

\subsection{Maximum a posteriori (MAP) estimate}
\frame{
\frametitle{Maximum a posteriori (MAP) estimate}

\begin{textblock}{14}(1, 4.5)
d

\end{textblock}

}

\subsection{Markov Random Fields (MRFs)}
\frame{blo
}

\subsection{Multiway Cut Problem}
\frame{blu
}


\section{Approximative Algorithms}
\subsection{$\alpha$-$\beta$ swap}
\subsection{$\alpha$-expansion}

\section{Application example}
\subsection{flute segmentation}



	
	
	
\end{document}